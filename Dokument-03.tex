\documentclass[ngerman]{scrreprt}

\usepackage[utf8]{inputenc}
\usepackage[T1]{fontenc}
\usepackage{babel}

\usepackage{listings}

\usepackage[]{xcolor}
\definecolor{hellgelb}{rgb}{1,1,0.8}
\definecolor{colKeys}{rgb}{0,0,1}
\definecolor{colIdentifier}{rgb}{0,0,0}
\definecolor{colComments}{rgb}{1,0,0}
\definecolor{colString}{rgb}{0,0.5,0}

\lstset{%
    float=hbp,%
    basicstyle=\ttfamily\small, %
    identifierstyle=\color{colIdentifier}, %
    keywordstyle=\color{colKeys}, %
    stringstyle=\color{colString}, %
    commentstyle=\color{colComments}, %
    columns=flexible, %
    tabsize=2, %
    frame=single, %
    extendedchars=true, %
    showspaces=false, %
    showstringspaces=false, %
    numbers=left, %
    numberstyle=\tiny, %
    breaklines=true, %
    backgroundcolor=\color{hellgelb}, %
    breakautoindent=true, %
    captionpos=b%
}

\usepackage{graphicx}

\usepackage{palatino}

\usepackage{scrpage2}
\pagestyle{scrheadings}

\setheadsepline[\textwidth]{1pt}{}
\setfootsepline[\textwidth]{1pt}{} 
 
\ohead{\headmark} % 
\ofoot{\pagemark} %  
\chead{central head}
\cfoot{central foot}
\ihead{inner head}
\ifoot{inner foot}

\author{Uwe Ziegenhagen}
\title{Mein erstes \LaTeX-Dokument}
\date{Oberhausen, \today}

\setlength{\parindent}{0pt}
\setlength{\parskip}{1em}


\begin{document}
\maketitle

\tableofcontents

\listoffigures

\listoftables

\chapter{Open Rhein Ruhr}

\section{Einleitung}\label{sec:einleitung}

The above links use the generic mirror.ctan.org url which autoredirects to a CTAN mirror that should be reasonably nearby and reasonably up to date. However, perfect synchronization is not possible; if you have troubles following the links, your best bet is to choose explicitly from the list of CTAN mirrors (you'll need to append systems/texlive/tlnet to the top-level mirror urls given there to get to the TL area). 

\begin{figure}
\begin{center}
\includegraphics{meinbild}
\caption{Hallo, ich bin ein Bild}\label{fig:meinbild}
\end{center}
\end{figure}

Siehe Abbildung \ref{fig:meinbild} auf Seite \pageref{fig:meinbild}

The above links use the generic mirror.ctan.org url which autoredirects to a CTAN mirror that should be reasonably nearby and reasonably up to date. However, perfect synchronization is not possible; if you have troubles following the links, your best bet is to choose explicitly from the list of CTAN mirrors (you'll need to append systems/texlive/tlnet to the top-level mirror urls given there to get to the TL area). 

The above links use the generic mirror.ctan.org url which autoredirects to a CTAN mirror that should be reasonably nearby and reasonably up to date. However, perfect synchronization is not possible; if you have troubles following the links, your best bet is to choose explicitly from the list of CTAN mirrors (you'll need to append systems/texlive/tlnet to the top-level mirror urls given there to get to the TL area). The above links use the generic mirror.ctan.org url which autoredirects to a CTAN mirror that should be reasonably nearby and reasonably up to date. However, perfect synchronization is not possible; if you have troubles following the links, your best bet is to choose explicitly from the list of CTAN mirrors (you'll need to append systems/texlive/tlnet to the top-level mirror urls given there to get to the TL area). The above links use the generic mirror.ctan.org url which autoredirects to a CTAN mirror that should be reasonably nearby and reasonably up to date. However, perfect synchronization is not possible; if you have troubles following the links, your best bet is to choose explicitly from the list of CTAN mirrors (you'll need to append systems/texlive/tlnet to the top-level mirror urls given there to get to the TL area). The above links use the generic mirror.ctan.org url which autoredirects to a CTAN mirror that should be reasonably nearby and reasonably up to date. However, perfect synchronization is not possible; if you have troubles following the links, your best bet is to choose explicitly from the list of CTAN mirrors (you'll need to append systems/texlive/tlnet to the top-level mirror urls given there to get to the TL area). The above links use the generic mirror.ctan.org url which autoredirects to a CTAN mirror that should be reasonably nearby and reasonably up to date. However, perfect synchronization is not possible; if you have troubles following the links, your best bet is to choose explicitly from the list of CTAN mirrors (you'll need to append systems/texlive/tlnet to the top-level mirror urls given there to get to the TL area). The above links use the generic mirror.ctan.org url which autoredirects to a CTAN mirror that should be reasonably nearby and reasonably up to date. However, perfect synchronization is not possible; if you have troubles following the links, your best bet is to choose explicitly from the list of CTAN mirrors (you'll need to append systems/texlive/tlnet to the top-level mirror urls given there to get to the TL area). The above links use the generic mirror.ctan.org url which autoredirects to a CTAN mirror that should be reasonably nearby and reasonably up to date. However, perfect synchronization is not possible; if you have troubles following the links, your best bet is to choose explicitly from the list of CTAN mirrors (you'll need to append systems/texlive/tlnet to the top-level mirror urls given there to get to the TL area). 

\section{Hauptteil}

Hallo, ich bin\footnote{Ich bin die erste Fußnote. However, perfect synchronization is not possible; if you have troubles following the links, your best bet is to choose explicitly from the list of CTAN mirrors (you'll need to append systems/texlive/tlnet to the top-level mirror urls given there to get to the TL area).} der Hauptteil.\footnote{Hallo, ich bin eine Fußnote.}

\subsection{Forschungsgrundlage}

\subsubsection{Literaturüberblick}

dsaihfois idshf is fhfdoi idsfoids oidshf oisifdso

\begin{table}
\caption{Ich bin eine Tabelle}
\begin{tabular}{lcrp{5cm}}
\bfseries Spalte 1 &\bfseries  Spalte 2 &\bfseries  Spalte 3 &\bfseries  Spalte 4 \\ \hline
Erste Spalte & 3231 & rechtsbündig & Hallo, ich bin ein Text \\ \hline
Erste Spalte & 3r534231 & noch rechtsbündig & Hallo, ich bin ein Text, ich hab euch lieb \\ \hline
\end{tabular}
\end{table}

\section{Source Code}

\lstinputlisting[language=Python,morekeywords={propagate}]{test.py}


\section{Fazit}

Zusammenfassend konnte in dieser Arbeit gezeigt werden, das es signifikante Unterschiede zwischen A und B gibt. Weitere Forschung wird genausowenig Resultate bringen.

\begin{equation}\label{eq:quad}
-\frac{p}{2} \pm \sqrt{ \left(\frac{p}{2}\right)^2  -q }
\end{equation}

Siehe Formel \ref{eq:quad}


Siehe Abschnitt \ref{sec:einleitung} auf Seite \pageref{sec:einleitung}

\include{kapitel2}

\end{document}

