\documentclass[ngerman]{scrartcl}

\usepackage[utf8]{inputenc}
\usepackage[T1]{fontenc}
\usepackage{babel}

\author{Uwe Ziegenhagen}
\title{Mein erstes \LaTeX-Dokument}
\date{Oberhausen, \today}

\setlength{\parindent}{0pt}
\setlength{\parskip}{1em}


\begin{document}
\maketitle

\tableofcontents

\section{Einleitung}

Hallo OpenRheinRuhr! öäüß

The above links use the generic mirror.ctan.org url which autoredirects to a CTAN mirror that should be reasonably nearby and reasonably up to date. However, perfect synchronization is not possible; if you have troubles following the links, your best bet is to choose explicitly from the list of CTAN mirrors (you'll need to append systems/texlive/tlnet to the top-level mirror urls given there to get to the TL area). 

The above links use the generic mirror.ctan.org url which autoredirects to a CTAN mirror that should be reasonably nearby and reasonably up to date. However, perfect synchronization is not possible; if you have troubles following the links, your best bet is to choose explicitly from the list of CTAN mirrors (you'll need to append systems/texlive/tlnet to the top-level mirror urls given there to get to the TL area). 

The above links use the generic mirror.ctan.org url which autoredirects to a CTAN mirror that should be reasonably nearby and reasonably up to date. However, perfect synchronization is not possible; if you have troubles following the links, your best bet is to choose explicitly from the list of CTAN mirrors (you'll need to append systems/texlive/tlnet to the top-level mirror urls given there to get to the TL area). The above links use the generic mirror.ctan.org url which autoredirects to a CTAN mirror that should be reasonably nearby and reasonably up to date. However, perfect synchronization is not possible; if you have troubles following the links, your best bet is to choose explicitly from the list of CTAN mirrors (you'll need to append systems/texlive/tlnet to the top-level mirror urls given there to get to the TL area). The above links use the generic mirror.ctan.org url which autoredirects to a CTAN mirror that should be reasonably nearby and reasonably up to date. However, perfect synchronization is not possible; if you have troubles following the links, your best bet is to choose explicitly from the list of CTAN mirrors (you'll need to append systems/texlive/tlnet to the top-level mirror urls given there to get to the TL area). The above links use the generic mirror.ctan.org url which autoredirects to a CTAN mirror that should be reasonably nearby and reasonably up to date. However, perfect synchronization is not possible; if you have troubles following the links, your best bet is to choose explicitly from the list of CTAN mirrors (you'll need to append systems/texlive/tlnet to the top-level mirror urls given there to get to the TL area). The above links use the generic mirror.ctan.org url which autoredirects to a CTAN mirror that should be reasonably nearby and reasonably up to date. However, perfect synchronization is not possible; if you have troubles following the links, your best bet is to choose explicitly from the list of CTAN mirrors (you'll need to append systems/texlive/tlnet to the top-level mirror urls given there to get to the TL area). The above links use the generic mirror.ctan.org url which autoredirects to a CTAN mirror that should be reasonably nearby and reasonably up to date. However, perfect synchronization is not possible; if you have troubles following the links, your best bet is to choose explicitly from the list of CTAN mirrors (you'll need to append systems/texlive/tlnet to the top-level mirror urls given there to get to the TL area). The above links use the generic mirror.ctan.org url which autoredirects to a CTAN mirror that should be reasonably nearby and reasonably up to date. However, perfect synchronization is not possible; if you have troubles following the links, your best bet is to choose explicitly from the list of CTAN mirrors (you'll need to append systems/texlive/tlnet to the top-level mirror urls given there to get to the TL area). 

\section{Hauptteil}

Hallo, ich bin der Hauptteil.

\subsection{Forschungsgrundlage}

\subsubsection{Literaturüberblick}

dsaihfois idshf is fhfdoi idsfoids oidshf oisifdso


\section{Fazit}

Zusammenfassend konnte in dieser Arbeit gezeigt werden, das es signifikante Unterschiede zwischen A und B gibt. Weitere Forschung wird genausowenig Resultate bringen.

\end{document}

